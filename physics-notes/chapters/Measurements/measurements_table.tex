\section{Measurements}
\paragraph{Changing units} Based on where we are in the world or what task we are trying to accomplish there exist different units of measure for the same quantity, a fundamental thing to know is how to switch between them: some changes are fairly trivial, like going from kilometer to meter ($\SI{1}{\kilo\metre} = \SI{E3}{\metre}$), but others not quite so- an example may be converting minutes to seconds or square kilometers to square miles.\\
The process is usually the same:
\begin{enumerate}
    \item Find/know the equivalence between two units of measure.
    \item Manipulate the ratio such that the wanted final unit is on top of the fraction.
    \item Apply the conversion.
\end{enumerate}
Following on the previous examples, our procedure would look like this:
\begin{itemize}
    \item $\SI{1}{\minute}=\SI{60}{\second}\to 1=\frac{\SI{60}{\second}}{\SI{1}{\minute}}$
    \begin{align*}
        t&=\SI{13}{\minute}\\
        &=1\times\SI{13}{\minute}\\
        &=\frac{\SI{60}{\second}}{\SI{1}{\minute}}\times\SI{13}{\minute} = \boxed{\SI{7.8E2}{\second}}
    \end{align*}
    \item $\SI{1.61}{\kilo\metre} = \SI{1}{\mathrm{mi}}\to 1=\frac{\SI{1}{\mathrm{mi}}}{\SI{1.61}{\kilo\metre}}\to \frac{\SI{1}{\mathrm{mi}^2}}{\SI{2.59}{\square\kilo\metre}}$
    \begin{align*}
        A&=\SI{27.0}{\square\kilo\metre}\\
        &=\frac{\SI{1}{\mathrm{mi}^2}}{\SI{2.59}{\square\kilo\metre}}\times\SI{27.0}{\square\kilo\metre}= \boxed{\SI{10.4}{\mathrm{mi}^2}}
    \end{align*}
\end{itemize}
\paragraph{Significant figures} The significant figures used to represent a quantity depend on the accuracy of the tool which took the survey: to count the amount of significant figures in a number just count all the digits which are \textbf{not} zero, all the zeroes (or groups of) which are in between non-zero figures and all of those zeroes which are deliberately left as decimal digits.\\
When displaying the result of a calculation, the number of significant figures to be chosen has to be equal to the lower amount of significant figures used by any value of the calculation.
\begin{equation*}
    1.22357894\times2.10 = 2.57
\end{equation*}