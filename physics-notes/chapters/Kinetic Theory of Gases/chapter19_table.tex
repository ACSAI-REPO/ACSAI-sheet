\documentclass{article}
\usepackage[utf8]{inputenc}
\usepackage{amsthm}
\usepackage{amsmath}
\usepackage{amssymb}
\usepackage{siunitx}
\usepackage{nicefrac}
\usepackage{tabularx}
\usepackage{graphicx}
\usepackage{wrapfig}
\usepackage{makecell}

%Done by Davide di Trocchio, 1940108, 5/12. 
%This table follows (Halliday: 19.1, 19.2, last part of 19.9) as written on the elearning by the professor. 
%It is based on benjamin's tables, just to have a standard to refer to. 
%Further formulas were not included to reduce confusion. If necessary just say so, i'll add them ASAP
%Also some of the equations do not have units, they need some fact-checking. ty <3
%Note that \end{tabularx} throws an error, idk why but it builds just fine so i didn't bother much. 

\begin{document}
\section{Kinetic Theory of Gases}
\paragraph{Basic Definitions}\

\begin{tabularx}{\textwidth}{l | X | l}
Quantity & Equation & Units \\
\hline \hline
 \hline
N_{A} (Avogadro's number) & \begin{equation*}
\num{6.02e23}  
\end{equation*}

 & mol^-^1  \\
 \hline

Mole & \begin{equation*}
    M = mN_A
    \end{equation*}
    
     & mol \\
     \hline

Number of moles & \begin{equation*}
    \emph{n} = \frac{\emph{N}}{N_A} = \frac{\emph{M\mathrm{sam}}}{M} = \frac{\emph{M\mathrm{sam}}}{mN_A}
    \end{equation*}
    
     & mol \\
     \hline

 Ideal Gas Law & \begin{equation*} 
    pV = nRT 
 \end{equation*} & \\
\hline

 Boltzmann constants' Ideal Gas Law & \begin{equation*}
    pV = NkT
\end{equation*} & \\
\hline

 k (Boltzmann constant) & \begin{equation*}
     P = \frac{R}{N_A} = \num{1.38e-23}
 \end{equation*} & $J/K$ \\
 \hline

 Slow Adiabatic Volume Change & \begin{equation*}
    pV^\gamma = a\:constant
\end{equation*} \
\begin{equation*}
    \gamma = \frac{C_p}{C_V}
\end{equation*}\\
\hline

Free Expansion & \begin{equation*}
    pV = a\:constant.
\end{equation*} \\
\hline

\end{tabularx}


\paragraph{Applications}\

\begin{tabularx}{\textwidth}{l | X }
Name & Equation \\
\hline\hline

Work done by an Ideal Gas & \begin{equation*}
    W = nRT\ln{\frac{V_f}{V_i}}
\end{equation*} & \\
\hline
\end{tabularx}

\end{document}