\documentclass{scrartcl}

\usepackage{amsthm}
\usepackage{amsmath}
\usepackage{amssymb}
\usepackage{siunitx}

\begin{document}
    \section{Motion in Two and Three Dimensions}
    \paragraph{Position Vector} It is a vector that extends from a reference point to a particle. In unit vector notation:
    \begin{equation}
        \vec{r} = x\hat{i}+y\hat{j}+z\hat{k}
    \end{equation}
    \paragraph{Displacement} As the particle moves, the position vector changes. The particle's displacement \vec{\Delta{r}} is:
    \begin{equation}
        \Delta\vec{r}= \vec{r_2} - \vec{r_1}
    \end{equation}
    Or, in unit vector notation:
    \begin{equation}
        \vec{\Delta{r}} = (x_2 - x_1) \hat{i} + (y_2 - y_1) \hat{j} + (z_2 - z_1) \hat{k}
    \end{equation}
    \paragraph{Average Velocity} When a particle moves through a displacement-r}}$ in a time interval ${\Delta{t}}$ , its average velocity is: 
    \begin{equation}
        \vec{v}_{avg} = \frac{\Delta\vec{r}}{\Delta{t}}
    \end{equation}
    The equation clarify that the direction of the velocity is the same as the direction of the displacement.
    In vector notion:
    \begin{equation}
        \vec{v}_{avg} = \frac{\Delta{x} \hat{i} + \Delta{y} \hat{j} + \Delta{z} \hat{k}}{\Delta{t}} = \frac{\Delta{x}}{\Delta{t}} \hat{i} + \frac{\Delta{y}}{\Delta{t}} \hat{j} + \frac{\Delta{z}}{\Delta{t}} \hat{k}
    \end{equation}
    \paragraph{Instantaneous Velocity} To find the velocity of a particle at instant \enhance{t} we take the value that $\vec{v}_{avg}$ assumes as the interval $\Delta{t}$ approaches to 0
    \begin{equation}
        \vec{x} = \frac{d\vec{r}}{dt}
    \end{equation}
    Or, in unit vector notation:
    \begin{equation}
        \vec{v} = v_x \hat{i} + v_y \hat{j} + v_z \hat{k}
    \end{equation}
    Where the scalar component are
    \begin{equation} 
        \newline v_x= \frac{dx}{dt} , v_y= \frac{dy}{dt} , v_z= \frac{dz}{dt}
    \end{equation}
    \paragraph{Average acceleration} When a particle's velocity changes, its average acceleration $\vec{a}_{avg}$ is
    \begin{equation}
        \vec{a}_{avg} = \frac{\vec{v}_2 - \vec{v}_1}{\Delta{t}} = \frac{\Delta{\vec{v}}}{\Delta{t}}
    \end{equation}
    \paragraph{Instantaneous Acceleration} As for the velocity, to know the acceleration in an instant \enhance{t}, we take the value of $\vec{a}_{avg}$ assumes as the interval $\Delta{t}$ approaches 0
    \begin{equation}
        \vec{a} = \frac{d\vec{v}}{dt}
    \end{equation}
    In unit vector notation:
    \begin{equation}
        \vec{a} = a_x \hat{i} + a_y \hat{j} + a_z \hat{k}
    \end{equation}
    Where the components of $\vec{a}$ are:
    \begin{equation}
        a_x = \frac{dv_x}{dt} , a_y = \frac{dv_y}{dt} , a_z = \frac{dv_z}{dt}
    \end{equation}
    The direction of an acceleration vectors does not extend from one position to another, it just show the direction for a particle located at its tail. 
    \paragraph{Projectile motion} A special case of two dimensional motion, the particle moves with a constant acceleration directed downwards, the free fall acceleration $\vec{g}$. When the projectile is launched, its initial velocity $\vec[v_0]$ is writable as: 
    \begin{equation}
        \vec{v_0} = v_{0x} \hat{i} + v_{0y} \hat{j}
    \end{equation}
    The two components $v_{0x}$ and $v_{0y}$ can be found if we know the angle $\theta_{0}$ between $\vec{v_0}$ and the positive x direction:
    \begin{equation}
        v_{0x} = v_0 \cos{\theta_0} , v_{0y} = v_0 \sin{\theta_0} 
    \end{equation}
    During the motion both position vector $\vec{r}$ and velocity vector $\vec{v}$ change continuously, though the acceleration remain constant. (There's no horizontal acceleration!)
    The two motions are independent of each other. The horizontal motion has no effect on the vertical one.
    
    \subparagraph{\small{Horizontal motion}} Since there is no acceleration, the horizontal component $v_x$ remains unchanged. Therefore at any time \enhance{t} the horizontal displacement $x - x_0$ is:
    \begin{equation}
        x - x_0 = v_{0x}t
    \end{equation}
    Since $v_{0x} = v_0 \cos{\theta_0}$ : 
    \begin{equation}
        x - x_0 = (v_0 \cos{\theta_0})t
    \end{equation}
    
    \subparagraph{\small{Vertical motion}} Since the acceleration is constant we can apply the equation of we have seen in the one dimensional motion chapter. Thus, we have the displacement $y- y_0$
    \begin{equation}
        y - y_0 = (v_{0} \sin{\theta_0})t - \frac{1}{2}gt^2
    \end{equation}
    
    Similarly, for the final velocity $v_y$
    \begin{equation}
        v_y = v_0 \sin{\theta_0} - gt 
    \end{equation}
    and
    \begin{equation}
        v^2_y = (v_0 \sin{\theta_0})^2 - 2g(y-y_0)
    \end{equation}
    \subparagraph{\small{The Equation of the path}} We can find the equation of the path of the projectile (called "trajectory" by solving equation (16) for t and substituting into equation (17).
    \begin{equation}
        y = (\tan{\theta_0})x - \frac{gx^2}{2(v_0 \cos{\theta_0})^2}
    \end{equation}
    \subparagraph{\small{The Horizontal Range}} The horizontal distance the projectile has traveled when it returns to its initial height is called Horizontal Range R
    \begin{equation}
        R = \frac{v^2_0}{g} \sin{\theta_0} \cos{\theta_0}
    \end{equation}
    The maximum horizontal range R is maximum at angle $\theta = 45^o$.
    
    Calculation done with the formula seen in this paragraph may differ a lot with the actual motion of the projectile as we assume the air has no effect on the projectile. 
    
    \paragraph{Uniform Circular Motion} A particle in uniform circular motion travels around a circle or a circular arc at constant speed. However, since the particle is moving in circles the direction of the velocity changes, thus the particle is accelerating. Velocity is always directed tangent to the circle in the direction of motion, while the acceleration is directed radially inward. Because of this, the acceleration is called centripetal acceleration. The magnitude of this acceleration $\vec{a}$ is:
    \begin{equation}
        a = \frac{v^2}{r}
    \end{equation}
    where r is the radius of the circle and v is the speed.
    
    During the motion, the particle travels the circumference of the circle in time T:
    \begin{equation}
        T = \frac{2\pi r}{v}
    \end{equation}
    T is called "period of revolution" or simply "period"
    
    \paragraph{Relative Motion in One Dimension} The velocity of a particle depends on the reference frame of whoever is observing it. Let's have two reference frames A and B, both observing a particle P, then the following equations hold:
    \newline
    Position:
    \begin{equation}
        x_{PA} = x_{PB} + x_{BA}
    \end{equation}
    To obtain the velocity equation we take the time derivative of equation (24)
    \newline
    Velocity:
    \begin{equation}
        v_{PA} = v_{PB} + v_{BA}
    \end{equation}
    To obtain the acceleration equation we take the time derivative of equation (25), since v_{BA} is constant the derivative will be 0.
    \newline
    Acceleration:
    \begin{equation}
        a_{PA} = a_{PB}
    \end{equation}
    Observers on different frames of reference that move at constant velocity relative to each other will measure the same acceleration for a moving particle. 
    \paragraph{Relative Motion in Two Dimensions} Similarly in two dimensions, having two reference frames A and B, both observing a particle P, the following equation holds:
    Position Vector:
    \begin{equation}
        \vec{r}_{PA} = \vec{r}_{PB} + \vec{r}_{BA}
    \end{equation}
    To obtain the velocity equation we take the time derivative of equation (27)
    \newline
    Velocity:
    \begin{equation}
        \vec{v}_{PA} = \vec{v}_{PB} + \vec{v}_{BA}
    \end{equation}
    To obtain the acceleration equation we take the time derivative of equation (25), since $v_{BA}$ is constant the derivative will be 0.
    \newline
    Acceleration:
    \begin{equation}
        a_{PA} = a_{PB}
    \end{equation}
    The same rule of One Dimensional motion holds, observers on different frames of reference that move at constant velocity relative to each other will measure the same acceleration for a moving particle. 

\end{document}
+++++++