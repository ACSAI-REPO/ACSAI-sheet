\documentclass{scrartcl} % This is the documentclass DONT-TOUCH-THIS

% These are the packages (library imports), if your chapter uses
% some package which is not here, feel free to add it (except for
% some special cases like the geometry package, for that ask on GitHub)
\usepackage{amsthm}
\usepackage{amsmath}
\usepackage{amssymb}
\usepackage{siunitx}
\usepackage{nicefrac}
\usepackage{tabularx}
\usepackage{graphicx}
\usepackage{wrapfig}

\newcommand{\tabeq}[1]{\parbox[c]{\hsize}{\begin{equation*}#1\end{equation*}}}

\sisetup{
    quotient-mode = fraction,
    per-mode = fraction,
    fraction-function=\nicefrac
}

% Defining the title of the doc.
\title{Physics' formulary}
\subtitle{by and for the Sapienza's ACSAI 2020/21 students}
\date{}

\begin{document}
\paragraph{Gravitation}\ 

\begin{tabularx}{\textwidth}{l | X}
    Notation & Equation\\
    \hline\hline
    Gravitational Force's Magnitude & \tabeq{
	{F} = {G}   \frac{m1 m2}{r^2}}\\
    \hline
    
    Gravitational Constant G & \tabeq{
	{G} = {6, 67 * 10^{-11} \frac{Nm}{kg^2}}\newline
	= {6, 67 * 10^{-11} \frac{m^3}{kg*s^2}}}\\
    \hline
    
    Principle of Superposition & \tabeq{
    \vec{F_{1,net}} = \vec{F_{1,2}}+ \vec{F_{1,3}} + ... + \vec{F_{1,n}}  =  \sum_{i=2}^{n} \vec{F_{1,i}}}\\
    \hline
    
    P.o.S on a Extended Real Object& \tabeq{
     \vec{F_{1}} = \int \!  \mathrm{d}\vec{F}}\\
    \hline
    
     Newton's Second Law& \tabeq{
     {F} =  {m}{a_{g}}}\\
    \hline
    
    Gravitational Acceleration& \tabeq{
     {a_{g}} =  \frac{GM}{r^2}}\\
    \hline
    
    Newton's Second Law for Forces along r axis& \tabeq{
    {F_N - ma_g} =  -m(\omega^2 R)}\\
      \hline
      
    Free-Fall Acceleration (Near Eearth's Surface)& \tabeq{
     {g} =  {}a_g - \omega^2 R}\\
    \hline
    
     Gravitational Force Inside Earth& \tabeq{
     {F} =  \frac{GmM}{R^3} r}\\
    \hline
    
    Gravitational Potential Energy 2-particles& \tabeq{
     {U} = -\frac{GMm}{r}}\\
    \hline
    
    Gravitational Potential Energy mul-particles& \tabeq{
      {U} = -(\frac{Gm_1m_2}{r_{1 2}} + \frac{Gm_1m_3}{r_{1 3}} + \frac{Gm_2m_3}{r_{2 3}}   ...)}\\
    \hline
    
    Change Gravitational Potential Energy(Path Indep.)& \tabeq{
    \Delta U = {U_f - U_i} = -W}\\
    \hline
    
    Escape Speed& \tabeq{
    {v} =  \sqrt{\frac{2GM}{R}}}\\
    \hline
    		    
\end{tabularx}
\end{document}

































