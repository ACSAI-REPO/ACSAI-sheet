\section{Gauss Law}
\paragraph{Gauss Law}\


\begin{tabularx}{\textwidth}{l | X}
     Law & Formula\\
     \hline\hline
     Gauss Law 
     & \tabeq {
        \epsilon_{0}\Phi = q_{enc}
     }\\
     \hline
     Flux Through\\Gaussian Surface
     & \tabeq {
        \Phi = \oint \vec{E}\cdot d\vec{A}
     }\\
      
\end{tabularx}\\

\paragraph{Applications}\

\begin{tabularx}{\textwidth}{l | X}

    Law & Formula\\
    \hline\hline
    conducting\\surface 
    & \tabeq{
        E = \frac{\sigma}{\epsilon_{0}}
    }\\
    \hline
    Line\\ of Charge 
    & \tabeq{
        E = \frac{\lambda}{2\pi\epsilon_{0}r}
    }\\
    \hline
     sheet\\ of charge 
     & \tabeq{
        E = \frac{\sigma}{2\epsilon_{0}}
     }\\
     \hline
     Spherical\\Shell
     & \tabeq{
        \frac{1}{4\pi\epsilon_{0}} \frac{q}{r^2}
     }
\end{tabularx}

\paragraph{Electric field in a uniform sphere of charge}\

\begin{tabularx}{\textwidth}{l | X}
    \hline
    Formula 
    & \tabeq{
        E = (\frac{q}{4\pi\epsilon_{0}R^3})r
    }
    
\end{tabularx}