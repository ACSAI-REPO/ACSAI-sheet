\documentclass{scrartcl} % This is the documentclass DONT-TOUCH-THIS


%   Rolling, Torque and Angular Momentum section - by Dario Loi
%   Document produced through refactoring (shameless theft)
%   of vectors_table.tex, and of my own force-and-motion.tex

%   remove unnecessary imports from the document when merging!
%   also please check for any duplicate formulas and the likes
%   since I ignore the bigger picture when writing these sections.


\usepackage{amsthm}
\usepackage{amsmath}
\usepackage{amssymb}
\usepackage{siunitx}
\usepackage{nicefrac}
\usepackage{tabularx}
\usepackage{graphicx}
\usepackage{wrapfig}

%Huge recommend to set this up in the whole document!~
%Makes fractions easier to see when in tables
\setlength{\extrarowheight}{8pt}

\newcommand{\tabeq}[1]{\parbox[c]{\hsize}{\begin{equation*}#1\end{equation*}}}

\sisetup{
    quotient-mode = fraction,
    per-mode = fraction,
    fraction-function=\nicefrac
}

\begin{document}
\section{Rolling, Torque, Angular Momentum}

\subsection{Rolling}
\paragraph{Quantities}\ 

\begin{tabularx}{\textwidth}{l | X | X}
    Quantity & Formula\\
    \hline\hline
    Rolling velocity
    & \tabeq{ 
        v_{com} = \omega
        }   \\
    \hline
    Rolling kinetic energy
    & \tabeq{ 
        K = \tfrac{1}{2}I_{com}\omega^2+\tfrac{1}{2}Mv^2_{com}
        }   \\
    \hline
    Linear acceleration of rolling body
    & \tabeq{ 
        a_{com,x} = -\cfrac{g\sin{\theta}}{1+I_{com}/MR^2}
        }   \\
    \hline

\end{tabularx}
\paragraph{Torque and Angular Momentum}\ 

\begin{tabularx}{\textwidth}{l | X}
    Quantity & Formula\\
    \hline\hline
    Torque
    & \tabeq{ 
        \Vec{\tau} = \Vec{r} \times \Vec{F}
        }   \\
    \hline
    Torque (magnitude)
    & \tabeq{ 
        \tau = rF\sin{\theta}
        }   \\
    \hline
    Angular Momentum
    & \tabeq{ 
        \textit{L} = \Vec{r} \times \Vec{p} \iff m(\Vec{r} \times \Vec{v})
        }   \\
    \hline
    Angular Momentum (magnitude)
    & \tabeq{ 
        \textit{L} = rmv\sin{\theta}
        }   \\
    \hline
    Second law (angular)
     & \tabeq{ 
        \vec{\tau_{net}} = \cfrac{ d\vec{\textit{L}} }{dt}
        }   \\
    \hline
    
\end{tabularx}

\end{document}