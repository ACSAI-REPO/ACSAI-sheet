\documentclass{scrartcl}

\usepackage{amsthm}
\usepackage{amsmath}
\usepackage{amssymb}
\usepackage{siunitx}
\usepackage{epigraph}
\usepackage{graphicx}
\usepackage{wrapfig}
\usepackage{nicefrac}
\usepackage{xcolor}

\begin{document}
    \section{Rotation}
	Rotation is the motion of a rigid\footnote{Parts composing it interlock together, they don't move: a piece of solid metal is ok, a container of fluids is not.} body about an axis; since this course is not advanced, an assumption which can be made is that the aforementioned axis won't itself move: that would make things extremely complex in no time (would be a mixture of rotation \textbf{and} translation).
    \subsection{Basic definitions} As we will see, most of rotation's rules and quantities follow the same \emph{patterns} as translation. Indeed, almost all the names are shared across the two types of motion.
    \paragraph{An object's angle} In everyday life angles are recorded and shared using degrees, these can be indeed very useful (also because we are used to them), but quite hard to manage when doing computations (which may also involve values which not defined as degrees): this is why radians may be a preferable way to record rotational values.
\end{document}