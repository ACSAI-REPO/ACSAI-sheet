\documentclass{scrartcl} % This is the documentclass DONT-TOUCH-THIS


%   Force and Motion I section - by Dario Loi
%   Document produced through refactoring (shameless theft)
%   of vectors_table.tex

%   remove unnecessary imports from the document when merging!


\usepackage{amsthm}
\usepackage{amsmath}
\usepackage{amssymb}
\usepackage{siunitx}
\usepackage{nicefrac}
\usepackage{tabularx}
\usepackage{graphicx}
\usepackage{wrapfig}

\newcommand{\tabeq}[1]{\parbox[c]{\hsize}{\begin{equation*}#1\end{equation*}}}

\sisetup{
    quotient-mode = fraction,
    per-mode = fraction,
    fraction-function=\nicefrac
}

% Defining the title of the doc.
\title{Physics' formulary}
\subtitle{by and for the Sapienza's ACSAI 2020/21 students}
\date{}

\begin{document}
\section{Force and Motion}

\subsection{Chapter I}
\paragraph{Units of Measurement}\ 

\begin{tabularx}{\textwidth}{l | X | X}
    Quantity & Unit & Formula\\
    \hline\hline
    Force 
    & \tabeq{ 
        [N] = Newton
        } 
    & \tabeq{
        N = Kg * \frac{m}{s^2}
        } \\
        

    \hline

\end{tabularx}
\paragraph{Newton's laws}\ 

\begin{tabularx}{\textwidth}{l | X}
    Law & States \\
    \hline\hline
    First law 
    & \tabeq{ 
        \vec{F}_{net} = 0 \iff v = const
        }  \\
    \hline
    Second law 
    & \tabeq{ 
        \vec{F}_{net} = m\vec{a}
        }  \\
    \hline
    Third law 
    & \tabeq{ 
        \vec{F}_{AB} = -\vec{F}_{BA}
        }  \\
    \hline
    
    
\end{tabularx}
\subsection{Chapter II}
\paragraph{Friction}\ 

\begin{tabularx}{\textwidth}{l | X}
    Type & Formula \\
    \hline\hline
    Static friction
    & \tabeq{ 
        f_{s,max} = \mu_s F_N
        }  \\
    \hline
    Kinetic friction
    & \tabeq{ 
        f_{k} = \mu_k F_N
        }  \\
    \hline
    
\end{tabularx}

\paragraph{Uniform Circular Motion}\ 

\begin{tabularx}{\textwidth}{l | X}
    Quantity & Formula \\
    \hline\hline
    Acceleration
    & \tabeq{ 
        a = \frac{v^2}{R}
        }  \\
    \hline
    Force
    & \tabeq{ 
        F = m\frac{v^2}{R}
        }  \\
    \hline
    
\end{tabularx}

\end{document}