\documentclass{article}
\usepackage[utf8]{inputenc}



\begin{document}
\section{Kinetic Energy and Work}
\paragraph{What is energy?}Energy is so broad that a clear definition is difficult to write. Technically, energy is a scalar quantity associated with the state (or condition) of one or more objects. For a looser definition, we could say that energy is the quantitative property that must be transferred to an object in order to perform work on, or to heat, the object.
\paragraph{Properties of energy}Energy can be transformed from one type to another, but the total amount is always the same (energy is conserved). No exception to this \emph{principle of energy conservation} has ever been found. 
\\\\
\textit{In this chapter we focus on only one type of energy \emph{(kinetic energy)} and on only one way in which energy can be transferred \emph{(work).}}
\subsection{Kinetic Energy}
\paragraph{Kinetic Energy.}Kinetic energy is associated with the \textit{state of motion} of an object. The faster the object moves, the greater is its kinetic energy. When the object is stationary, its kinetic energy is zero.
\\
For an object of mass $m$ whose speed $v$ is well below the speed of light, 
\begin{equation}
    \frac{1}{2}mv^2
\end{equation}
The SI unit of kinetic energy (and all types of energy) is the joule $(J)$, defined as
\begin{equation}
    1 joule = 1 J = 1 kg \cdot m^2/s^2.
\end{equation}
\subsection{Work and Kinetic Energy}
\paragraph{Work.}When we increase or decrease the kinetic energy, we account for these changes by saying that your force has transferred enegry \textit{to} the object from yourself or \textit{from} the object to yourself. In such a transfer, \textbf{work} $W$ is said to be \textit{done on the object by the force}. More formally, work can be defined as energy transferred to or from an object by means of a force acting on the object. Energy transferred to the object is positive work, and energy transferred from the object is negative work. 
\paragraph{An Expression for Work.} To calculate the work a force does on a object as the object moves through some displacement, we use only the force component along the object's displacement. The force component perpendicular to the displacement does zero work. 
\\After several calculations, we get the equation of work done by a constant force:
\begin{equation}
    W = Fd \cos\phi
\end{equation}
or
\begin{equation}
    W = \vec{F}^{} \cdot \vec{d}^{}
\end{equation}
where $\phi$ is the angle between the directions of the displacement $\vec{d}^{}$ and the force $\vec{F}^{}$ and F is the magnitude of $\vec{F}^{}$.
\paragraph{Cautions.}There are two restrictions to using these formulas: the force must be a \textit{constant force}, the object must be \textit{particle-like} (rigid).
\paragraph{Signs for Work.}The work done on an object by a force can be either positive or negative work. The sign of work depends of the angle $\phi$. A force does positive work when it has a vector component in the same direction as the displacement, and it does negative work when it has a vector component in the opposite direction. It does zero work when it has no such vector component.
\\
\textbf{Just for some help:}
\\
\begin{tabular}{ |c|c| } 
 \hline
 $\phi$ $<$ 90\textdegree & W $>$ 0 \\ 
 $\phi$ = 90\textdegree & W = 0 \\ 
 $\phi$ $>$ 90\textdegree & W $<$ 0  \\ 
 \hline
\end{tabular}
\paragraph{Net Work.}When two or more objects act on an object, the \textbf{net work} done on the object is the sum of the works done by individual forces. We can calculate the net work in two ways. (1) We can find the work done by each force and then sum those works. (2) Or we can first find the net force of those forces and then use Eq. 3 or Eq. 4.
\paragraph{Work-Kinetic Energy Theorem.}Let ${K_i}$ be the initial kinetic energy, ${K_f}$ - final kinetic energy, $\Delta K$ - the change in the kinetic energy of the object, and let $W$ be the net work done on it. Then,
\begin{equation}
    \Delta K = K_f - K_i = W,
\end{equation}
which means that\\(change in the kinetic energy of a particle = (net work done on a particle).
We can also write
\begin{equation}
    K_f = K_i + W,
\end{equation}
which means that\\(kinetic energy after the net work is done) = (kinetic energy before the net work) + (the net work done).\\These statements are known as the \textbf{work-kinetic energy theorem} for particles. They hold for both positive and negative work.
\subsection{Work Done by the Gravitational Force.}
\paragraph{Work Done by the Gravitational Force}We use Eq. 3 to express the work during a displacement $\vec{d}^{}$. For the force magnitude $F$, we use $mg$ as the magnitude of $\vec{F_g}^{}$. Thus, the work $W_g$ done by the gravitational force $\vec{F_g}^{}$ is
\begin{equation}
    W_g = mgd\cos \phi
\end{equation}
\paragraph{Work Done in Lifting and Lowering an Object.} When we lift a particle-like object by applying a vertical force $\vec{F}^{}$ to it, during the upward displacement, our applied force does positive work $W_a$ on the object, while the gravitational force does negative work $W_g$ on it. Our applied force tends to transfer energy to the object, while the gravitational force tends to transfer energy from it. By Eq. 5, the change $\Delta K$ in the kinetic energy of the object due to these two energy transfers is
\begin{equation}
    \Delta K = K_f - K_i = W_a + W_g,
\end{equation}
where $K_f$ is the kinetic energy at the end of the displacement and $K_i$ is that at the start of the displacement.
\subsection{Work Done by a Spring Force}
\paragraph{The Spring Force.}A spring force is obviously the force from a spring. A spring can be in three different conditions: it can be in its relaxed state, it can be stretched (extended) or compressed.\\The spring force is given by Hooke's law:
\begin{equation}
    \vec{F}^{} = -k\vec{d}^{},
\end{equation}
where $k$ is the spring constant (or force constant).\\For a spring that lies on the $x$ axis with the origin ($x = 0$), we can write Hooke's Law as follows:
\begin{equation}
    F_x = -kx
\end{equation}
\textbf{Note.}If $x$ is positive, $F_x$ is negative and vice versa.
\paragraph{The Work Done by a Spring Force.}When talking about the work, we have to make three assumptions about the spring. (1)It is \textit{massless}; (2)It is an \textit{ideal spring} (therefore, it obeys Hooke's Law exactly); (3) The contact between the block and the floor is \textit{frictionless}.\\Considering all these assumptions and doing a few calculations, we get the equation of \textbf{work done by a spring force}:
\begin{equation}
    W_s = \frac{1}{2}kx_i^2 - \frac{1}{2}kx_f^2
\end{equation}
\subsection{Work Done by a General Variable Force}
\paragraph{One Dimensional and Three Dimensional Analysis.}When the force $\vec{F}^{}$ on a particle-like object depends on the position of the object, the work done by $\vec{F}^{}$ on the object while the object moves from an initial position $r_i$ with coordinates $(x_i, y_i, z_i)$ to a final position $r_f$ with coordinates $(x_f, y_f, z_f)$ must be found by integrating the force. When all three components do not depend on each other(in three dimensional space), the work is
\begin{equation}
    W =  \int_{x_i}^{x_f} F_x \,dx + \int_{y_i}^{y_f} F_y \,dy + \int_{z_i}^{z_f} F_z \,dz
\end{equation}
In one dimensional space, the work is
\begin{equation}
    W = \int_{x_i}^{x_f} F (x) \,dx
\end{equation}
\subsection{Power}
\paragraph{Power.}The time rate at which work is done is said to be \textbf{power} due to the force. If a force does an amount of work $W$ in an amount of time $\Delta t$, the \textbf{average power} due to the force during that time interval is
\begin{equation}
    P_avg = \frac{W}{\Delta t}
\end{equation}
The \textbf{instantaneous power} $P$ is the instantaneous time rate of doing work, which we can write as
\begin{equation}
    P = \frac{dW}{dt}
\end{equation}
For a force $\vec{F}^{}$ at an angle $\phi$ to the direction of travel of the instantaneous velocity $\vec{v}^{}$, the instantaneous power is
\begin{equation}
    P = Fv\cos \phi = \vec{F}^{} \cdot \vec{v}^{}
\end{equation}
The \textbf{SI unit} of power is the joule per second. This unit is used so often that it has a special name, the \textbf{watt} $(W)$:\\
\begin{equation}
    1 watt = 1 W = 1 J/s = 0.738 ft \cdot lb/s
\end{equation}

\end{document}
